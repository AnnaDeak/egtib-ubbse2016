\section{A szimuláció eredményei}

A szimuláció pontosságának megvizsgálása érdekében a már rendelkezésünkre álló bemeneti paraméterekre(melyeket Archetti is használt\cite{archetti2013evolutionary}) elvégeztünk 100-100 szimulációt, minden egyes távolságra 1-től 5-ig.

\begin{multicols}{2}
	A bemeneti paraméterek a következőek voltak:
	\begin{itemize}[noitemsep]
		\item populáció mérete: 1000
		\item defektálók: 5\%
		\item generációk száma: 15
		\item kooperálók költsége: 0.01
		\item osztódás: nincs
	\end{itemize}
	A konstans függvények paraméterei:
	\begin{itemize}[noitemsep]
		\item $s = 2$
		\item $h = 1$
		\item $d = \frac{1}{2}D$
		\item $z = 20$
	\end{itemize}	
\end{multicols}

\subsection{A költség és nyereség hatása}
\subsection{Diffusion gradient hatása}
\subsection{Összegzés}

Összességében azt állíthatjuk, hogy sikerült egy olyan terméket létrehoznunk, mely megállja helyét
