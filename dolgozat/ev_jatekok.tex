\section{Evolúciós játékok}

\iffalse 
Általános bemutatás.
Összehasonlítás a "klasszikus" játékelmélettel - külön alcím/fejezet?
 racionális játékosok <-> nem racionális játékosok
 dinamikus - mit jelent?
\fi
\subsection{Klasszikus játékelméleti fogalmak}
A játékelmélet a matematika egyik interdiszciplináris ága, amely a különböző döntési vagy verseny helyzetekben résztvevő felek viselkedését tanulmányozza. Ezekben a \textit{játékokban} - többszereplős konfliktusos szituációkban - a választott stratégiák szorosan összefüggenek, mivel minden résztvevő döntése befolyásolja a többiek választásának eredményét is.\cite{wiki:gametheory} A \textit{játékosok} alapvető célja a nyereségük maximalizálása. A játékosok és a jutalom mibenléte tudományterületenként változó. A játékelmélet alkalmazható úgy a matematikában, közgazdaságtanban, mint a  szociológiában, biológiában felmerülő döntési problémák megoldására, modellezésére.

A játékokat különböző szempontok alapján több csoportra oszthatjuk. A játék lehet \textit{kooperatív}, ahol a játékosok külső hatásra koalíciókat alkotnak, vagy \textit{nem kooperatív}, ahol a résztvevők versenytársai egymásnak és ha létre is jön együttműködés, az önkéntes. 

\textit{Szimmetrikus} játékok esetén a haszon csak a választott stratégiától függ, a játékos személyétől nem. Ha a játékosok nem cserélhetőek fel anélkül, hogy a stratégiák nyereségén változtatnánk, \textit{aszimmetrikus} játékról beszélünk. A legismertebb kétszemélyes játékok, mint a későbbiekben tárgyalt fogolydilemma vagy héja-galamb játék, szimmetrikusak.

\subsubsection{Statikus és dinamikus játékok}
Az idő szerepe fontos osztályozási kritérium, eszerint beszélhetünk \textit{statikus} és \textit{dinamikus} játékokról. Statikus játékok legfontosabb tulajdonsága, hogy a játékosok már a játék elején, egymástól függetlenül döntenek. Dinamikus játékok esetén számít a lépések sorrendje, mivel a játékosok ismerik a többiek eddigi lépéseit. A utóbbi csoportba sorolják az ismételt játékokat is, amikor egy statikus játékot véges vagy végtelenszer megismételnek. Itt a résztvevők minden kör után megfigyelhetik a többiek eddigi lépéseit és az alapján választhatják meg a következő körben alkalmazott stratégiát. A két típust különbözőképpen ábrázoljuk. A nem-kooperatív statikus játékokat általában egy mátrix segítségével írjuk le, amelyről leolvasható az összes lehetséges kombináció eredménye. Ezt nevezzük \textit{normál alaknak}. Dinamikus játék esetén, az ún. \textit{extenzív alakban}, az egymás utáni lépéseket egy véges irányított fa ábrázolja, amely gyökere a kezdőállapot, ahonnan a fa minden pontjához pontosan egy úton lehet eljutni.

\subsubsection{Nash-egyensúly}
A játékelmélet egyik központi fogalma a Nash-egyensúly. Egy stratégia-együttes Nash-egyensúlyban van ha, a csoportban egyik félnek sem érdemes eltérni az alkalmazott stratégiától, amennyiben egyik fél sem változtat a sajátján. 

dinamikák: szinkron aszinkron

\subsection{Játékelmélet a biológiában}
A játékelmélet alkalmazása a biológiában az \textit{evolúciós játékelmélet}. Maynard Smith, a terület első és legismertebb alakja, kijelenti, hogy az addigi nézetekkel ellentétben, a játékosoknak nem kell kötelező módon racionálisan dönteniük, elegendő ha van egy stratégiájuk. A stratégia genetikailag örökölt jelleg, ami az egyed viselkedését befolyásolja. Sikerét az mutatja, hogy mennyire tud fennmaradni és milyen gyakorisággal jelenik meg más stratégiák között. A résztvevők szelekciós sikerét a \textit{fitnesz} fejezi ki (a szaporodásra való relatív esély). A klasszikus játékelmélettel ellentétben itt a stratégiák változásának dinamikájára helyeződik/tevődik a hangsúly. A egyensúlyt itt az \textit{evolúciósan stabil stratégia} jelenti. Egy stratégia evolúciósan stabil, ha az azt alkalmazó populáció egy adott környezetben nem győzhető le semmilyen más alternatív, kezdetben kevés létszámú stratégiával.

\subsubsection{Replikátor dinamika}
itt lesznek képletek és minden, hogy érthető legyen, hogy hogyan is történik
\subsubsection{Adaptív dinamika}
\subsubsection{Szinkron és aszinkron}
\subsubsection{Térbeli játékok}
\iffalse
Kezdetben a darwini kiválasztódást próbálták modellezni a játékelmélet eszközeinek segítségével. 


Játékok típusai, osztályozás
Evol. játékok mint egy ág

\subsection{Evolúciós stabil stratégia -- fitnesz mint nyereség}
\subsection{Játékok dinamikája(Dynamic game theory)}
\subsubsection{Replikátor dinamika}
\subsubsection{Adaptív dinamika (Adaptive dynamics)}
\subsection{Térbeli játékok (spatial games)}

\fi

