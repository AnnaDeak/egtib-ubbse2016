\section{Evolúciós játékok}

\iffalse 
Általános bemutatás.
Összehasonlítás a "klasszikus" játékelmélettel - külön alcím/fejezet?
 racionális játékosok <-> nem racionális játékosok
 dinamikus - mit jelent?
\fi
\subsection{Klasszikus játékelméleti fogalmak}
A játékelmélet a matematika egyik interdiszciplináris ága, amely a különböző döntési vagy verseny helyzetekben résztvevő felek viselkedését tanulmányozza. Ezekben a \textit{játékokban} - többszereplős konfliktusos szituációkban - a választott stratégiák szorosan összefüggenek, mivel minden résztvevő döntése befolyásolja a többiek választásának eredményét is.\cite{wiki:gametheory} A \textit{játékosok} alapvető célja a nyereségük maximalizálása. A játékosok és a jutalom mibenléte tudományterületenként változó. A játékelmélet alkalmazható úgy a matematikában, közgazdaságtanban, mint a  szociológiában, biológiában felmerülő döntési problémák megoldására, modellezésére.

A játékokat különböző szempontok alapján több csoportra oszthatjuk. A játék lehet \textit{kooperatív}, ahol a játékosok külső hatásra koalíciókat alkotnak, vagy \textit{nem kooperatív}, ahol a résztvevők versenytársai egymásnak és ha létre is jön együttműködés, az önkéntes. 

\textit{Szimmetrikus} játékok esetén a haszon csak a választott stratégiától függ, a játékos személyétől nem. Ha a játékosok nem cserélhetőek fel anélkül, hogy a stratégiák nyereségén változtatnánk, \textit{aszimmetrikus} játékról beszélünk. A legismertebb kétszemélyes játékok, mint a későbbiekben tárgyalt fogolydilemma vagy héja-galamb játék, szimmetrikusak.

\subsubsection{Statikus és dinamikus játékok}
Az idő szerepe fontos osztályozási kritérium, eszerint beszélhetünk \textit{statikus} és \textit{dinamikus} játékokról. Statikus játékok legfontosabb tulajdonsága, hogy a játékosok már a játék elején, egymástól függetlenül döntenek. Dinamikus játékok esetén számít a lépések sorrendje, mivel a játékosok ismerik a többiek eddigi lépéseit. A utóbbi csoportba sorolják az ismételt játékokat is, amikor egy statikus játékot véges vagy végtelenszer megismételnek. Itt a résztvevők minden kör után megfigyelhetik a többiek eddigi lépéseit és az alapján választhatják meg a következő körben alkalmazott stratégiát. A két típust különbözőképpen ábrázoljuk. A nem-kooperatív statikus játékokat általában egy mátrix segítségével írjuk le, amelyről leolvasható az összes lehetséges kombináció eredménye. Ezt nevezzük \textit{normál alaknak}. Dinamikus játék esetén, az ún. \textit{extenzív alakban}, az egymás utáni lépéseket egy véges irányított fa ábrázolja, amely gyökere a kezdőállapot, ahonnan a fa minden pontjához pontosan egy úton lehet eljutni.

\subsubsection{Nash-egyensúly}
A játékelmélet egyik központi fogalma a Nash-egyensúly. Egy stratégia-együttes Nash-egyensúlyban van ha a csoportban egyik félnek sem érdemes eltérni az alkalmazott stratégiától, amennyiben egyik fél sem változtat a sajátján. Tekintsük \(G = ((N,S_i,u_i), i = 1,...,n)\) rendszert egy véges stratégia játék reprezentációjának, ahol \textit{N} a játékosok halmaza, \textit{n} pedig a játékosok száma. Minden \(i \in N\) játékos a \(S_i\) lépés-halmazból választhat. \(S = S_1 \times S_2 \times ... \times S_n\)  a játék összes kimenetelének halmaza, ahol \(s \in S\) egy stratégia-együttes. Minden \(i \in N\) játékos nyereségét a \(u_i:S \to \mathbb{R}\) hasznosságfüggvény írja le. Jelöljük \((s_i^*,s_i) = (s^*_1,...,s_i,...,s_n^*)\) -vel azt a stratégia-együttest, amelyet az \(s^*\)-ból kapunk ha az \textit{i} játékos stratégiáját kicseréljük \(s_i\)-re. Egy \(s^*\) stratégia-együttes Nash-egyensúlyban van, ha az \(u_i(s_i^*,s_i)\leq u_i(s^*) \) egyenlőtlenség igaz \(\forall i = 1,...,n, \forall s_i \in S_i, s_i \ne s_i^*\) esetén \cite{nash1951non}.

dinamikák: szinkron aszinkron

\subsection{Játékelmélet a biológiában}
A játékelmélet alkalmazása a biológiában az \textit{evolúciós játékelmélet}. Maynard Smith, a terület első és legismertebb alakja, kijelenti, hogy az addigi nézetekkel ellentétben, a játékosoknak nem kell kötelező módon racionálisan dönteniük, elegendő ha van egy stratégiájuk. A stratégia genetikailag örökölt jelleg, ami az egyed viselkedését befolyásolja. Sikerét az mutatja, hogy mennyire tud fennmaradni és milyen gyakorisággal jelenik meg más stratégiák között. A résztvevők szelekciós sikerét a \textit{fitnesz} fejezi ki (a szaporodásra való relatív esély). Az evolúciós játékok segítségével modellezhetjük a fajon belüli rivalizálást, a ragadozó-préda kapcsolatot de akár a gének vagy baktériumok közötti kölcsönhatásokat is. A klasszikus játékelmélettel ellentétben itt a stratégiák változásának dinamikájára helyeződik/tevődik a hangsúly. Egészen egyszerű játékok esetén is megtörténik, hogy a hosszútávú eredmény nem a Nash-egyensúly, hanem rendszeres vagy rendszertelen oszcillálás \cite{nowak2004evolutionary}. Evolúciós játékokban az egyensúlyt az \textit{evolúciósan stabil stratégia} jelenti. Egy stratégia evolúciósan stabil, ha az azt alkalmazó populáció egy adott környezetben nem győzhető le semmilyen más alternatív, kezdetben kevés létszámú stratégiával.

- nemek aránya, fajon belüli rivalizálás, ragadozó-préda kapcsolat stb
- gének, baktériumok
- Nash egyensúly nem elég , oszcillálás , 
Egyik legegyszerűbb játékelméleti modell két stratégiával és négy lehetséges eredménnyel dolgozik. A játékosok együttműködhetnek vagy defektálhatnak. Ha mindkét játékos kooperál nagyon a nyereségük mintha mindketten defektálnának, de ha egyik kooperál a másiknak jobban megéri kihasználni azt és nem együttműködni. A kooperáló játékossal két dolog történhet: vagy az éri meg jobban ha ő is változtat a stratégián és nem hagyja, hogy kizsákmányolják, mint a jól ismert fogolydilemmában (ábra), vagy még mindig akkor dönt jobban ha továbbra is együttműködik, ahogyan a héja-galamb játékban (ábra) történik. A kihívást a nyereségfüggvény helyes beállítása jelenti, hiszen attól függ, hogy egyik vagy másik játékhoz hasonlítjuk a természetben előforduló jelenségeket. 
legegyszerűbb modell: két stratégia, négy lehetséges eredmény: kooperál, defektál
lehetséges esetek tárgyalni: kimenet: fogolydilemma vagy héja galamb
kihivás: melyik játékhoz hasonlítjuk? a hasznossági függvény jó beállítása
általánosan két stratégia esetén: a eltűnik stb

\subsubsection{Replikátor dinamika}
evolúciós játékok dinamikája
dinamika differenciálegyenlete 
gyakoriság-függő szelekció biológiai játékokban
heterogén pop, véges számú stratégia, növekedés arányos a fitnesszel
ez a dinamika átalakul, ha térbeli játékokkal foglalkozunk
itt legtöbbször rácsszerűen helyezkednek el és csak a szomszédokkal lépnek kapcsolatba
szinkron aszinkron

\subsubsection{Adaptív dinamika}
replikátor  nem beszél mutációról
végeredmény függ a mutáció mértékétől
képlet
evolúciósan stabil - nem lehet győzheti le mutáns ha a kezdeti arány elég kicsi
ha a jelenlegi veri asz összeset: verhetetlen


\iffalse
Kezdetben a darwini kiválasztódást próbálták modellezni a játékelmélet eszközeinek segítségével. 


Játékok típusai, osztályozás
Evol. játékok mint egy ág

\subsection{Evolúciós stabil stratégia -- fitnesz mint nyereség}
\subsection{Játékok dinamikája(Dynamic game theory)}
\subsubsection{Replikátor dinamika}
\subsubsection{Adaptív dinamika (Adaptive dynamics)}
\subsection{Térbeli játékok (spatial games)}

\fi
