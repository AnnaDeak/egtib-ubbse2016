\section{Összegzés}

Elmondhatjuk, hogy sikerült egy olyan szoftvert létrehoznunk mely segítségével jobban megérhetjük, hogy mi is történik egy daganaton belül. Fő erőssége a kísérletezésben rejlik, de kutatási vagy akár tanítási célokra is fel lehet használni. Webes alkalmazás révén egyszerűen elérhető, bárki által használható. Letisztult felülete átláthatóvá és könnyen használhatóvá teszi. Tudtunk szerint, a játékelmélet ezen területén, ez az első önálló szoftver, amely 
\begin{itemize}[noitemsep]
	\item interaktívan mutatja be a sejtpopulációk változását
	\item ezen bemutatáshoz voronoi diagramot használva
	\item nem igényel programozási tudás, hiszen képes szimulációkat végezni pár kattintással
\end{itemize}

Természetesen hiányosságai is vannak, melyeket a jövőben szeretnénk pótolni. Rendkívül hasznos lenne azon funkció mely során egy terápiát (pl. kemoterápia) alkalmazunk, vagy ellenanyagokat juttatunk be, melyek a termelést, annak költségét befolyásolják. Komoly kihívásnak számít az, hogy különböző ráktípusokra meghatározzuk az őket leíró paramétereket. A számítások hatékonyságágának növelése is egy fontos feladat, ami maga után vonná azt a tényt, hogy nagyobb populációkkal és diffúziós gradienssel is szimulálhatnánk. 
Jelenleg csak papíron létezik több fajta stratégia (ref arra a 4 képre ami a stratégiákat mutatja) de a közeljövőben be szeretnénk vezetni még legalább egyet mint válaszható opciót.

Az általunk végzett szimulációk arra engednek következtetni, hogy a sejtek játéka bizonyos határokon belül leírja a daganatok viselkedését és hisszük azt, hogy a játékelmélettel ezen területet ki lehet aknázni.