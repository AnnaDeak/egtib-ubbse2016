\chapter{Bevezető}

Dolgozatunk témája az evolúciós játékok alkalmazása daganatos sejtek viselkedésének modellezésében. A játékelmélet mint tudományág kevesebb mint egy évszázados múltra tekint vissza, mégis kutatók ezreit, köztük nem egy Nobel-díjas tudóst foglalkoztat különböző tudományterületeken. Biológiai alkalmazása, ezen belül a daganatos sejtek tanulmányozása egészen új területnek számít. Dolgozatunk elméleti háttereként a témában írt legfrissebb tudományos munkák szolgáltak.

Az első fejezetben bevezetjük a későbbiekben használt játékelméleti fogalmakat és a játékok különböző szempontok szerinti osztályozását. Bemutatjuk azt, hogy hol és hogyan alkalmazhatóak ezek a fogalmak a biológiában, megismerkedünk az evolúciós játékelmélet alapjaival és az ezt mozgató dinamikákkal. Az evolúciós játékok segítségével modellezhetjük a fajon belüli rivalizálást, a ragadozó-préda kapcsolatot de akár a gének vagy baktériumok közötti kölcsönhatásokat is. Minket jelen esetben nem az egyedek viselkedése érdekel, hanem az, ami sejtek szintjén történik.

A következő fejezetben nagyon röviden tárgyaljuk a daganatos sejtek viselkedését, majd az itt alkalmazható játékelméleti modelleket. Ismertetjük az általunk használt modellt, annak szereplőit és "játékszabályait". Bevezetjük a Voronoi diagramokat és elmondjuk, hogy milyen változtatásokat eszközöltünk mi az eredeti modellhez képest. 

Ezután következik az alkalmazás részletes bemutatása. A projekt gyakorlati részét egy webes felület képezi, amelyen megjeleníthetőek a fent bemutatott modellek, pontosabban a sejtek viselkedése, a populáció időbeli alakulása. Részletezzük az alkalmazás funkcionalitásait és felépítését, illetve a felhasznált technológiákat.

Végül összegezzük a szimulációnk eredményét. Megvizsgáltuk a különböző paraméterek változtatásának a populáció dinamikájára gyakorolt hatását, az eredeti elméleti kutatás eredményeit, majd az eredeti és a mi bővített modellünk közötti eltéréseket.

Mivel kezdetben a munka egy csoportos projektként indult, ezúton szeretnénk megköszönni a segítséget a csapat többi tagjának.